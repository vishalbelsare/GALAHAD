\begin{description}

\itt{status} is a scalar variable of type \integer, that gives the
exit status of the algorithm. 
%See Sections~\ref{galerrors} and \ref{galinfo}
See Section~\ref{galerrors} 
for details.

\itt{alloc\_status} is a scalar variable of type \integer, that gives
the status of the last attempted array allocation or deallocation.
This will be 0 if {\tt status = 0}.

\itt{bad\_alloc} is a scalar variable of type default \character\
and length 80, that  gives the name of the last internal array 
for which there were allocation or deallocation errors.
This will be the null string if {\tt status = 0}. 

\ittf{iter} is a scalar variable of type \integer, that gives the
total number of iterations required.

\itt{factorization\_status} is a scalar variable of type \integer, that 
gives the return status from the matrix factorization.

\itt{factorization\_integer} is a scalar variable of type long
\integer, that gives the amount of integer storage used for the matrix 
factorization.

\itt{factorization\_real} is a scalar variable of type \longinteger, 
that gives the amount of real storage used for the matrix factorization.

\itt{nfacts} is a scalar variable of type \integer, that gives the
total number of factorizations performed.

\itt{nbacts} is a scalar variable of type \integer, that gives the
total number of backtracks performed during the sequence of linesearches.

\itt{nmods} is a scalar variable of type \integer, that gives the
total number of factorizations which were modified to 
ensure that the matrix is an appropriate preconditioner. 

\ittf{obj} is a scalar variable of type \realdp, that holds the
value of the objective function at the best estimate of the solution found.

\itt{non\_negligible\_pivot} is a scalar variable of type \realdp, 
that holds the value of the smallest pivot larger than {\tt control\%zero\_pivot}
when searching for dependent linear constraints. If 
{\tt non\_negligible\_pivot} is close to  {\tt control\%zero\_pivot},
this may indicate that there are further dependent constraints, and
it may be worth increasing {\tt control\%zero\_pivot} above 
{\tt non\_negligible\_pivot} and solving again.

\itt{feasible} is a scalar variable of type default \logical, that has the
value \true\ if the output value of $\bmx$ satisfies the constraints,
and the value \false\ otherwise.

\ittf{time} is a scalar variable of type {\tt \packagename\_time\_type} 
whose components are used to hold elapsed CPU  and system clock 
times for the various parts of the calculation (see Section~\ref{typetime}).

\itt{LSQP\_inform} is a scalar variable of type 
{\tt LSQP\_inform\_type}
whose components are used to provide information about the 
initial feasible point calculation
performed by the package 
{\tt \libraryname\_LSQP}. 
See the specification sheet for the package 
{\tt \libraryname\_LSQP} 
for details, and appropriate default values.

\itt{FDC\_inform} is a scalar variable of type 
{\tt FDC\_inform\_type}
whose components are used to provide information about 
any detection of linear dependencies
performed by the package 
{\tt \libraryname\_FDC}. 
See the specification sheet for the package 
{\tt \libraryname\_FDC} for details.

\itt{SBLS\_inform} is a scalar variable of type 
{\tt SBLS\_inform\_type}
whose components are used to provide information about factorizations
performed by the package 
{\tt \libraryname\_SBLS}. 
See the specification sheet for the package 
{\tt \libraryname\_SBLS} for details.

\itt{GLTR\_inform} is a scalar variable of type 
{\tt GLTR\_inform\_type}
whose components are used to provide information about the step calculation
performed by the package 
{\tt \libraryname\_GLTR}. 
See the specification sheet for the package 
{\tt \libraryname\_GLTR} 
for details.

\end{description}
