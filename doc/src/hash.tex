\documentclass{galahad}

% set the package name

\newcommand{\package}{hash}
\newcommand{\packagename}{HASH}
\newcommand{\fullpackagename}{\libraryname\_\packagename}
\newcommand{\solver}{{\tt \fullpackagename\_solve}}

\begin{document}

\galheader

%%%%%%%%%%%%%%%%%%%%%% SUMMARY %%%%%%%%%%%%%%%%%%%%%%%%

\galsummary
This package {\bf sets up a dictionary of words, allowing a user to insert new
words, and search for and remove existing words}.  It also allows the user
to rebuild the dictionary if the maximum allowed word-size, or the
total space provided, proves too small.  Provided sufficient room is
allowed, the expected number of operations required for an insertion,
search or removal is $O(1)$.  The method is based on the chained
scatter table insertion method of F. A. Williams.

%%%%%%%%%%%%%%%%%%%%%% attributes %%%%%%%%%%%%%%%%%%%%%%%%

\galattributes
\galversions{\tt  \fullpackagename\_single, \fullpackagename\_double}.
\galuses
{\tt GALAHAD\_SY\-M\-BOLS},
{\tt GALAHAD\-\_SPECFILE}
and
{\tt GALAHAD\_SPACE}.
\galdate July 2021.
\galorigin N. I. M. Gould, Rutherford Appleton Laboratory.
\gallanguage Fortran~95 + TR 15581 or Fortran~2003.

%%%%%%%%%%%%%%%%%%%%%% HOW TO USE %%%%%%%%%%%%%%%%%%%%%%%%

\galhowto

The package is available with single, double and (if available) 
quadruple precision reals, and either 32-bit or 64-bit integers. 
Access to the 32-bit integer,
single precision version requires the {\tt USE} statement
\medskip

\hspace{8mm} {\tt USE \fullpackagename\_single}

\medskip
\noindent
with the obvious substitution 
{\tt \fullpackagename\_double},
{\tt \fullpackagename\_quadruple},
{\tt \fullpackagename\_single\_64},
{\tt \fullpackagename\_double\_64} and
{\tt \fullpackagename\_quadruple\_64} for the other variants.


\noindent
If it is required to use more than one of the modules at the same time, 
the derived types
{\tt \packagename\_time\_\-type},
{\tt \packagename\_control\_type},
{\tt \packagename\_inform\_type},
{\tt \packagename\_data\_type}
and
(Section~\ref{galtypes})
and the subroutines
{\tt \packagename\_initialize},
{\tt \packagename\_insert},
{\tt \packagename\-\_search},
{\tt \packagename\_remove},
{\tt \packagename\_rebuild}
{\tt \packagename\_terminate},
(Section~\ref{galarguments})
and
{\tt \packagename\_read\_specfile}
(Section~\ref{galfeatures})
must be renamed on one of the {\tt USE} statements.

%%%%%%%%%%%%%%%%%%%%%%%%%%% kinds %%%%%%%%%%%%%%%%%%%%%%%%%%%

%%%%%%%%%%%%%%%%%%%%%% kinds %%%%%%%%%%%%%%%%%%%%%%

\galkinds
We use the terms integer and real to refer to the fortran
keywords {\tt REAL(rp\_)} and {\tt INTEGER(ip\_)}, where 
{\tt rp\_} and {\tt ip\_} are the relevant kind values for the
real and integer types employed by the particular module in use. 
The former are equivalent to
default {\tt REAL} for the single precision versions,
{\tt DOUBLE PRECISION} for the double precision cases
and quadruple-precision if 128-bit reals are available, and 
correspond to {\tt rp\_ = real32}, {\tt rp\_ = real64} and
{\tt rp\_ = real128} respectively as defined by the 
fortran {\tt iso\_fortran\_env} module.
The latter are default (32-bit) and long (64-bit) integers, and
correspond to {\tt ip\_ = int32} and {\tt ip\_ = int64},
respectively, again from the {\tt iso\_fortran\_env} module.


%%%%%%%%%%%%%%%%%%%%%% derived types %%%%%%%%%%%%%%%%%%%%%%%%

\galtypes
Three derived data types are accessible from the package.

%%%%%%%%%%% control type %%%%%%%%%%%

\subsubsection{The derived data type for holding control
 parameters}\label{typecontrol}
The derived data type
{\tt \packagename\_control\_type}
is used to hold controlling data. Default values may be obtained by calling
{\tt \packagename\_initialize}
(see Section~\ref{subinit}),
while components may also be changed by calling
{\tt \fullpackagename\_read\-\_spec}
(see Section~\ref{readspec}).
The components of
{\tt \packagename\_control\_type}
are:

\begin{description}

\itt{error} is a scalar variable of type \integer, that holds the
stream number for error messages. Printing of error messages in
{\tt \packagename\_solve} and {\tt \packagename\_terminate}
is suppressed if {\tt error} $\leq 0$.
The default is {\tt error = 6}.

\ittf{out} is a scalar variable of type \integer, that holds the
stream number for informational messages. Printing of informational messages in
{\tt \packagename\_solve} is suppressed if {\tt out} $< 0$.
The default is {\tt out = 6}.

\itt{print\_level} is a scalar variable of type \integer, that is used
to control the amount of informational output which is required. No
informational output will occur if {\tt print\_level} $\leq 0$. If
{\tt print\_level} $= 1$, a single line of output will be produced for each
iteration of the process. If {\tt print\_level} $\geq 2$, this output will be
increased to provide significant detail of each iteration.
The default is {\tt print\_level = 0}.

\itt{space\_critical} is a scalar variable of type default \logical,
that must be set \true\ if space is critical when allocating arrays
and  \false\ otherwise. The package may run faster if
{\tt space\_critical} is \false\ but at the possible expense of a larger
storage requirement. The default is {\tt space\_critical = .FALSE.}.

\itt{deallocate\_error\_fatal} is a scalar variable of type default \logical,
that must be set \true\ if the user wishes to terminate execution if
a deallocation  fails, and \false\ if an attempt to continue
will be made. The default is {\tt deallocate\_error\_fatal = .FALSE.}.

\itt{prefix} is a scalar variable of type default \character\
and length 30, that may be used to provide a user-selected
character string to preface every line of printed output.
Specifically, each line of output will be prefaced by the string
{\tt prefix(2:LEN(TRIM( prefix ))-1)},
thus ignoreing the first and last non-null components of the
supplied string. If the user does not want to preface lines by such
a string, they may use the default {\tt prefix = ""}.

\end{description}

%%%%%%%%%%% inform type %%%%%%%%%%%

\subsubsection{The derived data type for holding informational
 parameters}\label{typeinform}
The derived data type
{\tt \packagename\_inform\_type}
is used to hold parameters that give information about the progress and needs
of the algorithm. The components of
{\tt \packagename\_inform\_type}
are:

\begin{description}
\itt{status} is a scalar variable of type \integer, that gives the
exit status of the algorithm.
See Section~\ref{galerrors} for details.

\itt{alloc\_status} is a scalar variable of type \integer, that gives
the status of the last attempted array allocation or deallocation.
This will be 0 if {\tt status = 0}.

\itt{bad\_alloc} is a scalar variable of type default \character\
and length 80, that  gives the name of the last internal array
for which there were allocation or deallocation errors.
This will be the null string if {\tt status = 0}.

\end{description}

%%%%%%%%%%% data type %%%%%%%%%%%

\subsubsection{The derived data type for holding problem data}\label{typedata}
The derived data type
{\tt \packagename\_data\_type}
is used to hold all the data associated with the dictionary built
and maintained between calls of {\tt \packagename} procedures.
This data should be preserved, untouched 
from the initial call to
{\tt \packagename\_initialize}
to the final call to
{\tt \packagename\_terminate}.

%%%%%%%%%%%%%%%%%%%%%% argument lists %%%%%%%%%%%%%%%%%%%%%%%%

\galarguments
All the words in the dictionary are entered into a so-called
{\em chained scatter table}. Before the first word is entered, the
table must be initialized by a call to {\tt \packagename\_initialize}
to set default values, and initialize private data.
Words are inserted in the table by calling {\tt \packagename\_insert}. 
The table may be searched for an existing word with a call to 
{\tt \packagename\_search}; an existing word may be flagged as deleted
from the table by calling {\tt \packagename\_remove}. 
Finally, the table may be rebuilt to
allow for an increase in the maximum allowed word-size or the total
number of entries in the table with a call to {\tt \packagename\_rebuild}.
All internally allocated workspace may be removed by calling
{\tt \packagename\_terminate}.

%%%%%% initialization subroutine %%%%%%

\subsubsection{The initialization subroutine}\label{subinit}
Dictionary initialization and default control parameters 
are provided as follows:
\vspace*{1mm}

\hspace{8mm}
{\tt CALL \packagename\_initialize( nchar, length, data, control, inform )}

\vspace*{-2mm}
\begin{description}

\ittf{nchar} is a scalar \intentin\ argument of type \integer\ that
should be sent to an upper bound on the number of characters in each
word that may be inserted into the dictionary.

\itt{length} is a scalar \intentin\ argument of type \integer\ that
should be sent to an upper bound on the number of words that
may be inserted into the dictionary.

\itt{data} is a scalar \intentinout\ argument of type
{\tt \packagename\_data\_type}
(see Section~\ref{typedata}). It is used to hold data about the dictionary.

\itt{control} is a scalar \intentout\ argument of type
{\tt \packagename\_control\_type}
(see Section~\ref{typecontrol}).
On exit, {\tt control} contains default values for the components as
described in Section~\ref{typecontrol}.
These values should only be changed after calling
{\tt \packagename\_initialize}.

\itt{inform} is a scalar \intentout\ argument of type
{\tt \packagename\_inform\_type}
(see Section~\ref{typeinform}). A successful call to
{\tt \packagename\_initialize}
is indicated when the  component {\tt status} has the value 0.
For other return values of {\tt status}, see Section~\ref{galerrors}.

\end{description}

%%%%%% insert subroutine %%%%%%

\subsubsection{The insertion subroutine}\label{subinsert}
A word may be inserted into the dictionary as follows:
\vspace*{1mm}

\hspace{8mm}
{\tt CALL \packagename\_insert( nchar, field, position, data, control, inform )}

\vspace*{-2mm}
\begin{description}

\ittf{nchar} is a scalar \intentin\ argument of type \integer\ that
should be sent to an upper bound on the number of characters in each
word that may be inserted into the dictionary.

\itt{field} is an array \intentin\ argument of length {\tt nchar} and 
type default \character\ that contains the characters of the word that 
is to be inserted into the dictionary.  
Component {\tt field(j)} should be filled with the {\tt j}-th 
character of the word. If the word contains fewer than {\tt nchar} characters,
it should be padded with blanks.

\itt{position} is a scalar \intentout\ argument of type \integer\ that
gives the index of the table that data for the word occupies after insertion.
If {\tt position=0} on exit, there is no more room in the dictionary, and
it should be rebuilt (see \S\ref{subrebuild}) with more space before trying 
the insertion again.

\itt{data} is a scalar \intentinout\ argument of type
{\tt \packagename\_data\_type}
(see Section~\ref{typedata}). It is used to hold data about the dictionary.

\itt{control} is a scalar \intentin\ argument of type
{\tt \packagename\_control\_type}
(see Section~\ref{typecontrol}).
On exit, {\tt control} contains default values for the components as
described in Section~\ref{typecontrol}.
These values should only be changed after calling
{\tt \packagename\_initialize}.

\itt{inform} is a scalar \intentinout\ argument of type
{\tt \packagename\_inform\_type}
(see Section~\ref{typeinform}). A successful call to
{\tt \packagename\_initialize}
is indicated when the  component {\tt status} has the value 0.
For other return values of {\tt status}, see Section~\ref{galerrors}.

\end{description}

%%%%%% search subroutine %%%%%%

\subsubsection{The search subroutine}\label{subsearch}
A word may be searched for in the dictionary as follows:
\vspace*{1mm}

\hspace{8mm}
{\tt CALL \packagename\_search( nchar, field, position, data, control, inform )}

\vspace*{-2mm}
\begin{description}

\ittf{nchar} is a scalar \intentin\ argument of type \integer\ that
should be sent to an upper bound on the number of characters in each
word that may be inserted into the dictionary.

\itt{field} is an array \intentin\ argument of length {\tt nchar} and 
type default \character\ that contains the characters of the word that 
is to be searched for in the dictionary.  
Component {\tt field(j)} should be filled with the {\tt j}-th 
character of the word. If the word contains fewer than {\tt nchar} characters,
it should be padded with blanks.

\itt{position} is a scalar \intentout\ argument of type \integer\ that
gives the index of the table that data for the word occupies. If the word is
not found, {\tt position} will be 0, and if the word has been removed, it will
be negative (and {\tt - position} was the index that it once occupied).

\itt{data} is a scalar \intentinout\ argument of type
{\tt \packagename\_data\_type}
(see Section~\ref{typedata}). It is used to hold data about the dictionary.

\itt{control} is a scalar \intentin\ argument of type
{\tt \packagename\_control\_type}
(see Section~\ref{typecontrol}).
On exit, {\tt control} contains default values for the components as
described in Section~\ref{typecontrol}.
These values should only be changed after calling
{\tt \packagename\_initialize}.

\itt{inform} is a scalar \intentinout\ argument of type
{\tt \packagename\_inform\_type}
(see Section~\ref{typeinform}). A successful call to
{\tt \packagename\_initialize}
is indicated when the  component {\tt status} has the value 0.
For other return values of {\tt status}, see Section~\ref{galerrors}.

\end{description}

%%%%%% remove subroutine %%%%%%

\subsubsection{The removal subroutine}\label{subremove}
A word may be removed from the dictionary as follows:
\vspace*{1mm}

\hspace{8mm}
{\tt CALL \packagename\_remove( nchar, field, position, data, control, inform )}

\vspace*{-2mm}
\begin{description}

\ittf{nchar} is a scalar \intentin\ argument of type \integer\ that
should be sent to an upper bound on the number of characters in each
word that may be inserted into the dictionary.

\itt{field} is an array \intentin\ argument of length {\tt nchar} and 
type default \character\ that contains the characters of the word that 
is to be inserted into the dictionary.  
Component {\tt field(j)} should be filled with the {\tt j}-th 
character of the word. If the word contains fewer than {\tt nchar} characters,
it should be padded with blanks.

\ittf{position} is a scalar \intentout\ argument of type \integer\ that
gives the index of the table that data for the word occupies before it was
removed. If the word is not found, {\tt position} will be 0.

\itt{data} is a scalar \intentinout\ argument of type
{\tt \packagename\_data\_type}
(see Section~\ref{typedata}). It is used to hold data about the dictionary.

\itt{control} is a scalar \intentin\ argument of type
{\tt \packagename\_control\_type}
(see Section~\ref{typecontrol}).
On exit, {\tt control} contains default values for the components as
described in Section~\ref{typecontrol}.
These values should only be changed after calling
{\tt \packagename\_initialize}.

\itt{inform} is a scalar \intentinout\ argument of type
{\tt \packagename\_inform\_type}
(see Section~\ref{typeinform}). A successful call to
{\tt \packagename\_initialize}
is indicated when the  component {\tt status} has the value 0.
For other return values of {\tt status}, see Section~\ref{galerrors}.

\end{description}

%%%%%% rebuild subroutine %%%%%%

\subsubsection{The rebuilding subroutine}\label{subrebuild}
The dictionary may be rebuilt to increase in its length as follows:
\vspace*{1mm}

\hspace{8mm}
{\tt CALL \packagename\_rebuild( length, new\_length, moved\_to, data, control, inform )}

\vspace*{-2mm}
\begin{description}

\itt{length} is a scalar \intentin\ argument of type \integer\ that
should be the current length of the dictionary.

\itt{new\_length} is a scalar \intentin\ argument of type \integer\ 
that should be the new length of the dictionary.

\itt{moved\_to} is an array \intentout\ argument of length {\tt length} and 
type \integer\ that gives the position in the new table
that the old table entry has been moved to. Specifically, if 
{\tt moved\_to(i)} is nonzero, entry {\tt i} has moved to position
{\tt moved\_to(i)} in the new dictionary, while if {\tt moved\_to(i)=0},
entry {\tt i} was not previously occupied.

\itt{data} is a scalar \intentinout\ argument of type
{\tt \packagename\_data\_type}
(see Section~\ref{typedata}). It is used to hold data about the dictionary.

\itt{control} is a scalar \intentin\ argument of type
{\tt \packagename\_control\_type}
(see Section~\ref{typecontrol}).
On exit, {\tt control} contains default values for the components as
described in Section~\ref{typecontrol}.
These values should only be changed after calling
{\tt \packagename\_initialize}.

\itt{inform} is a scalar \intentinout\ argument of type
{\tt \packagename\_inform\_type}
(see Section~\ref{typeinform}). A successful call to
{\tt \packagename\_initialize}
is indicated when the  component {\tt status} has the value 0.
For other return values of {\tt status}, see Section~\ref{galerrors}.

\end{description}

%%%%%%% termination subroutine %%%%%%

\subsubsection{The  termination subroutine}
All previously allocated arrays are deallocated as follows:
\vspace*{1mm}

\hspace{8mm}
{\tt CALL \packagename\_terminate( data, control, inform )}

\vspace*{-1mm}
\begin{description}

\itt{data} is a scalar \intentinout\ argument of type
{\tt \packagename\_data\_type}
exactly as for
{\tt \packagename\_solve},
which must not have been altered {\bf by the user} since the last call to
{\tt \packagename\_initialize}.
On exit, array components will have been deallocated.

\itt{control} is a scalar \intentin\ argument of type
{\tt \packagename\_control\_type}
exactly as for
{\tt \packagename\_solve}.

\itt{inform} is a scalar \intentout\ argument of type
{\tt \packagename\_inform\_type}
exactly as for
{\tt \packagename\_solve}.
Only the component {\tt status} will be set on exit, and a
successful call to
{\tt \packagename\_terminate}
is indicated when this  component {\tt status} has the value 0.
For other return values of {\tt status}, see Section~\ref{galerrors}.

\end{description}

%%%%%%%%%%%%%%%%%%%%%% Warning and error messages %%%%%%%%%%%%%%%%%%%%%%%%

\galerrors
A negative value of {\tt inform\%status} on exit from
{\tt \packagename\_solve}
or
{\tt \packagename\_terminate}
indicates that an error has occurred. No further calls should be made
until the error has been corrected. Possible values are:

\begin{description}

\itt{\galerrallocate.} A workspace allocation error occurred.
A message indicating the offending
array is written on unit {\tt control\%error}, and the returned allocation
status and a string containing the name of the offending array
are held in {\tt inform\%alloc\_\-status}
and {\tt inform\%bad\_alloc}, respectively.

\itt{\galerrdeallocate.} A workspace deallocation error occurred.
A message indicating the offending
array is written on unit {\tt control\%error} and the returned allocation
status and a string containing the name of the offending array
are held in {\tt inform\%alloc\_\-status}
and {\tt inform\%bad\_alloc}, respectively.

\itt{-99.} The current dictionary is full and should be rebuilt with more space
 (see \S\ref{subrebuild}).

\end{description}

%%%%%%%%%%%%%%%%%%%%%% Further features %%%%%%%%%%%%%%%%%%%%%%%%

\galfeatures
\noindent In this section, we describe an alternative means of setting
control parameters, that is components of the variable {\tt control} of type
{\tt \packagename\_control\_type}
(see Section~\ref{typecontrol}),
by reading an appropriate data specification file using the
subroutine {\tt \packagename\_read\_specfile}. This facility
is useful as it allows a user to change  {\tt \packagename} control parameters
without editing and recompiling programs that call {\tt \packagename}.

A specification file, or specfile, is a data file containing a number of
"specification commands". Each command occurs on a separate line,
and comprises a "keyword",
which is a string (in a close-to-natural language) used to identify a
control parameter, and
an (optional) "value", which defines the value to be assigned to the given
control parameter. All keywords and values are case insensitive,
keywords may be preceded by one or more blanks but
values must not contain blanks, and
each value must be separated from its keyword by at least one blank.
Values must not contain more than 30 characters, and
each line of the specfile is limited to 80 characters,
including the blanks separating keyword and value.

The portion of the specification file used by
{\tt \packagename\_read\_specfile}
must start
with a "{\tt BEGIN \packagename}" command and end with an
"{\tt END}" command.  The syntax of the specfile is thus defined as follows:
\begin{verbatim}
  ( .. lines ignored by HASH_read_specfile .. )
    BEGIN HASH
       keyword    value
       .......    .....
       keyword    value
    END
  ( .. lines ignored by HASH_read_specfile .. )
\end{verbatim}
where keyword and value are two strings separated by (at least) one blank.
The ``{\tt BEGIN \packagename}'' and ``{\tt END}'' delimiter command lines
may contain additional (trailing) strings so long as such strings are
separated by one or more blanks, so that lines such as
\begin{verbatim}
    BEGIN HASH SPECIFICATION
\end{verbatim}
and
\begin{verbatim}
    END HASH SPECIFICATION
\end{verbatim}
are acceptable. Furthermore,
between the
``{\tt BEGIN \packagename}'' and ``{\tt END}'' delimiters,
specification commands may occur in any order.  Blank lines and
lines whose first non-blank character is {\tt !} or {\tt *} are ignored.
The content
of a line after a {\tt !} or {\tt *} character is also
ignored (as is the {\tt !} or {\tt *}
character itself). This provides an easy manner to "comment out" some
specification commands, or to comment specific values
of certain control parameters.

The value of a control parameters may be of three different types, namely
integer, logical or real.
Integer and real values may be expressed in any relevant Fortran integer and
floating-point formats (respectively). Permitted values for logical
parameters are "{\tt ON}", "{\tt TRUE}", "{\tt .TRUE.}", "{\tt T}",
"{\tt YES}", "{\tt Y}", or "{\tt OFF}", "{\tt NO}",
"{\tt N}", "{\tt FALSE}", "{\tt .FALSE.}" and "{\tt F}".
Empty values are also allowed for
logical control parameters, and are interpreted as "{\tt TRUE}".

The specification file must be open for
input when {\tt \packagename\_read\_specfile}
is called, and the associated device number
passed to the routine in device (see below).
Note that the corresponding
file is {\tt REWIND}ed, which makes it possible to combine the specifications
for more than one program/routine.  For the same reason, the file is not
closed by {\tt \packagename\_read\_specfile}.

\subsubsection{To read control parameters from a specification file}
\label{readspec}

Control parameters may be read from a file as follows:
\hskip0.5in
\def\baselinestretch{0.8} {\tt
\begin{verbatim}
     CALL HASH_read_specfile( control, device )
\end{verbatim}
}
\def\baselinestretch{1.0}

\begin{description}
\itt{control} is a scalar \intentinout argument of type
{\tt \packagename\_control\_type}
(see Section~\ref{typecontrol}).
Default values should have already been set, perhaps by calling
{\tt \packagename\_initialize}.
On exit, individual components of {\tt control} may have been changed
according to the commands found in the specfile. Specfile commands and
the component (see Section~\ref{typecontrol}) of {\tt control}
that each affects are given in Table~\vref{specfile}.

\bctable{|l|l|l|}
\hline
  command & component of {\tt control} & value type \\
\hline
  {\tt error-printout-device} & {\tt \%error} & integer \\
  {\tt printout-device} & {\tt \%out} & integer \\
  {\tt print-level} & {\tt \%print\_level} & integer \\
  {\tt space-critical}   & {\tt \%space\_critical} & logical \\
  {\tt deallocate-error-fatal}   & {\tt \%deallocate\_error\_fatal} & logical \\
\hline

\ectable{\label{specfile}Specfile commands and associated
components of {\tt control}.}

\itt{device} is a scalar \intentin argument of type \integer,
that must be set to the unit number on which the specfile
has been opened. If {\tt device} is not open, {\tt control} will
not be altered and execution will continue, but an error message
will be printed on unit {\tt control\%error}.

\end{description}

%%%%%%%%%%%%%%%%%%%%%% Information printed %%%%%%%%%%%%%%%%%%%%%%%%

\galinfo
If {\tt control\%print\_level} is positive, basic information about the 
progress of the algorithm will be printed on unit {\tt control\-\%out}.

%%%%%%%%%%%%%%%%%%%%%% GENERAL INFORMATION %%%%%%%%%%%%%%%%%%%%%%%%

\galgeneral

\galcommon None.
\galworkspace Provided automatically by the module.
\galroutines None.
\galmodules {\tt \packagename\_solve} calls the \galahad\ packages
{\tt GALAHAD\_SY\-M\-BOLS},
{\tt GALAHAD\_SPECFILE}
and
{\tt GALAHAD\_SPACE}.
\galio Output is under control of the arguments
 {\tt control\%error}, {\tt control\%out} and {\tt control\%print\_level}.
\galrestrictions None.
\galportability ISO Fortran~95 + TR 15581 or Fortran~2003.
The package is thread-safe.

%%%%%%%%%%%%%%%%%%%%%% METHOD %%%%%%%%%%%%%%%%%%%%%%%%

\galmethod

To insert a word in the table, the word is first mapped onto an
integer value, the entry integer.  This mapping is often called
hashing. As many words may be mapped to the same value (a collision),
a chain of used locations starting from the entry integer is searched
until an empty location is found. The word is inserted in the table at
this point and the chain extended to the next unoccupied entry. The
hashing routine is intended to reduce the number of collisions.  Words
are located and flagged as deleted from the table in exactly the same
way; the word is hashed and the resulting chain searched until the
word is matched or the end of the chain reached.  Provided there is
sufficient space in the table, the expected number of operations
needed to perform an insertion, search or removal is $O(1)$.

\galreferences
\vspace*{1mm}

\noindent
The chained scatter table search and insertion method is due to 
\vspace*{1mm}

\noindent
F. A.  Williams (1959),
``Handling identifies as internal symbols in language processors'',
Communications of the ACM {\bf 2(6)} pp 21-24.

%%%%%%%%%%%%%%%%%%%%%% EXAMPLE %%%%%%%%%%%%%%%%%%%%%%%%

\galexamples
As a simple example, we read a set of words into a table and then
test if a second set of words is present. Each set is terminated by a
blank.  A maximum word length of 10 characters is specified.  We may
use the following code.

{\tt \small
\VerbatimInput{\packageexample}
}
\noindent
The code produces the following output:
{\tt \small
\VerbatimInput{\packageresults}
}

\end{document}
