\documentclass{galahad}

% set the package name

\newcommand{\package}{fdh}
\newcommand{\packagename}{FDH}
\newcommand{\fullpackagename}{\libraryname\_\packagename}
\newcommand{\solver}{{\tt \fullpackagename\_analyse}}

\begin{document}

\galheader

%%%%%%%%%%%%%%%%%%%%%% SUMMARY %%%%%%%%%%%%%%%%%%%%%%%%

\galsummary
This package 
{\bf computes a finite-difference approximation to the Hessian matrix} 
$\bmH(\bmx)$, for which  
$(\bmH(\bmx))_{i,j} = \partial f^2 / \partial x_i \partial x_j$,
$1 \leq i, j \leq n$,  
using values of the gradient $\bmg(\bmx) = \nabla_x f(\bmx)$ 
of the function $f(\bmx)$ of $n$ unknowns $\bmx = (x_1, \ldots, x_n)^T$. 
The method takes advantage of the entries in the Hessian that are known to
be zero.  The user must specify the step sizes to be used in the
finite difference calculation and either supply a routine to evaluate
the gradient or provide gradient values by reverse communication.

%%%%%%%%%%%%%%%%%%%%%% attributes %%%%%%%%%%%%%%%%%%%%%%%%

\galattributes
\galversions{\tt  \fullpackagename\_single, \fullpackagename\_double}.
\galuses 
{\tt GALAHAD\_SY\-M\-BOLS}, 
{\tt GALAHAD\_SP\-ECFILE},
{\tt GALAHAD\_SPACE} and
{\tt GALAHAD\_NLPT}.
\galdate July 2012.
\galorigin N. I. M. Gould, Rutherford Appleton Laboratory,
and Ph. L. Toint, The University of Namur, Belgium.
\gallanguage Fortran~95 + TR 15581 or Fortran~2003. 

%%%%%%%%%%%%%%%%%%%%%% HOW TO USE %%%%%%%%%%%%%%%%%%%%%%%%

\galhowto

The package is available with single, double and (if available) 
quadruple precision reals, and either 32-bit or 64-bit integers. 
Access to the 32-bit integer,
single precision version requires the {\tt USE} statement
\medskip

\hspace{8mm} {\tt USE \fullpackagename\_single}

\medskip
\noindent
with the obvious substitution 
{\tt \fullpackagename\_double},
{\tt \fullpackagename\_quadruple},
{\tt \fullpackagename\_single\_64},
{\tt \fullpackagename\_double\_64} and
{\tt \fullpackagename\_quadruple\_64} for the other variants.


The user is {\bf strongly advised} to use the double
precision version unless single precision corresponds to 8-byte arithmetic.

\noindent
If it is required to use more than one of the modules at the same time, 
the derived types
{\tt \packagename\_control\_type}, 
{\tt \packagename\_inform\_type},
{\tt \packagename\_data\_type}
and
{\tt NLPT\_userdata\_type},
(Section~\ref{galtypes})
and the subroutines
{\tt \packagename\_initialize}, 
{\tt \packagename\_\-analyse},
{\tt \packagename\_\-estimate},
{\tt \packagename\_terminate},
(Section~\ref{galarguments})
and 
{\tt \packagename\_read\_specfile}
(Section~\ref{galfeatures})
must be renamed on one of the {\tt USE} statements.

%%%%%%%%%%%%%%%%%%%%%%%%%%% kinds %%%%%%%%%%%%%%%%%%%%%%%%%%%

%%%%%%%%%%%%%%%%%%%%%% kinds %%%%%%%%%%%%%%%%%%%%%%

\galkinds
We use the terms integer and real to refer to the fortran
keywords {\tt REAL(rp\_)} and {\tt INTEGER(ip\_)}, where 
{\tt rp\_} and {\tt ip\_} are the relevant kind values for the
real and integer types employed by the particular module in use. 
The former are equivalent to
default {\tt REAL} for the single precision versions,
{\tt DOUBLE PRECISION} for the double precision cases
and quadruple-precision if 128-bit reals are available, and 
correspond to {\tt rp\_ = real32}, {\tt rp\_ = real64} and
{\tt rp\_ = real128} respectively as defined by the 
fortran {\tt iso\_fortran\_env} module.
The latter are default (32-bit) and long (64-bit) integers, and
correspond to {\tt ip\_ = int32} and {\tt ip\_ = int64},
respectively, again from the {\tt iso\_fortran\_env} module.


%%%%%%%%%%%%%%%%%%%%%% derived types %%%%%%%%%%%%%%%%%%%%%%%%

\galtypes
Four derived data types are accessible from the package.

%%%%%%%%%%% control type %%%%%%%%%%%

\subsubsection{The derived data type for holding control 
 parameters}\label{typecontrol}
The derived data type 
{\tt \packagename\_control\_type} 
is used to hold controlling data. Default values may be obtained by calling 
{\tt \packagename\_initialize}
(see Section~\ref{subinit}),
while components may also be changed by calling 
{\tt \fullpackagename\_read\-\_spec}
(see Section~\ref{readspec}). 
The components of 
{\tt \packagename\_control\_type} 
are:


\begin{description}

\itt{error} is a scalar variable of type \integer, that holds the
stream number for error messages. Printing of error messages in 
{\tt \packagename\_analyse},
{\tt \packagename\_estimate} 
and {\tt \packagename\_terminate} 
is suppressed if {\tt error} $\leq 0$.
The default is {\tt error = 6}.

\ittf{out} is a scalar variable of type \integer, that holds the
stream number for informational messages. Printing of informational messages in 
{\tt \packagename\_analyse} and {\tt \packagename\_estimate} 
is suppressed if {\tt out} $< 0$.
The default is {\tt out = 6}.

\itt{print\_level} is a scalar variable of type \integer, that is used
to control the amount of informational output which is required. No 
informational output will occur if {\tt print\_level} $\leq 0$. If 
{\tt print\_level} $> 01$, details of any data errors encountered 
will be reported.
The default is {\tt print\_level = 0}.

\itt{space\_critical} is a scalar variable of type default \logical, 
that must be set \true\ if space is critical when allocating arrays
and  \false\ otherwise. The package may run faster if 
{\tt space\_critical} is \false\ but at the possible expense of a larger
storage requirement. The default is {\tt space\_critical = .FALSE.}.

\itt{deallocate\_error\_fatal} is a scalar variable of type default \logical, 
that must be set \true\ if the user wishes to terminate execution if
a deallocation  fails, and \false\ if an attempt to continue
will be made. The default is {\tt deallocate\_error\_fatal = .FALSE.}.

\itt{prefix} is a scalar variable of type default \character\
and length 30, that may be used to provide a user-selected 
character string to preface every line of printed output. 
Specifically, each line of output will be prefaced by the string 
{\tt prefix(2:LEN(TRIM( prefix ))-1)},
thus ignoreing the first and last non-null components of the
supplied string. If the user does not want to preface lines by such
a string, they may use the default {\tt prefix = ""}.

\end{description}

%%%%%%%%%%% inform type %%%%%%%%%%%

\subsubsection{The derived data type for holding informational
 parameters}\label{typeinform}
The derived data type 
{\tt \packagename\_inform\_type} 
is used to hold parameters that give information about the progress and needs 
of the algorithm. The components of 
{\tt \packagename\_inform\_type} 
are:

\begin{description}
\itt{status} is a scalar variable of type \integer, that gives the
exit status of the algorithm. 
See Sections~\ref{reverse} and \ref{galerrors}
for details.

\itt{alloc\_status} is a scalar variable of type \integer, that gives
the status of the last attempted array allocation or deallocation.
This will be 0 if {\tt status = 0}.

\itt{bad\_alloc} is a scalar variable of type default \character\
and length 80, that  gives the name of the last internal array 
for which there were allocation or deallocation errors.
This will be the null string if {\tt status = 0}. 

\itt{bad\_row} is a scalar variable of type \integer, that holds the 
index of the first row in which inconsistent data occurred (or 0 if the data is 
consistent).

\itt{products} is a scalar variable of type \integer, that gives the
number of gradient evaluations (to be) used to estimate the Hessian.

\end{description}

%%%%%%%%%%% data type %%%%%%%%%%%

\subsubsection{The derived data type for holding problem data}\label{typedata}
The derived data type 
{\tt \packagename\_data\_type} 
is used to hold all the data for a particular problem,
or sequences of problems with the same structure, between calls of 
{\tt \packagename} procedures. 
This data should be preserved, untouched (except as directed on
return from \solver\ with positive values of {\tt inform\%status}, see
Section~\ref{reverse}),
from the initial call to 
{\tt \packagename\_initialize}
to the final call to
{\tt \packagename\_terminate}.

%%%%%%%%%%% userdata type %%%%%%%%%%%

\subsubsection{The derived data type for holding user data}\label{typeuserdata}
The derived data type 
{\tt NLPT\_userdata\_type} 
is available to allow the user to pass data to and from user-supplied 
subroutines for function and derivative calculations (see Section~\ref{gfv}).
Components of variables of type {\tt NLPT\_userdata\_\-type} may be allocated as
necessary. The following components are available:

\begin{description}
\itt{integer} is a rank-one allocatable array of type \integer.
\ittf{real} is a rank-one allocatable array of type default  \realdp
\itt{complex} is a rank-one allocatable array of type default \complexdp.
\itt{character} is a rank-one allocatable array of type default \character.
\itt{logical} is a rank-one allocatable array of type default \logical.
\itt{integer\_pointer} is a rank-one pointer array of type \integer.
\itt{real\_pointer} is a rank-one pointer array of type default  \realdp
\itt{complex\_pointer} is a rank-one pointer array of type default \complexdp.
\itt{character\_pointer} is a rank-one pointer array of type default \character.
\itt{logical\_pointer} is a rank-one pointer array of type default \logical.
\end{description}

%%%%%%%%%%%%%%%%%%%%%% argument lists %%%%%%%%%%%%%%%%%%%%%%%%

\galarguments
There are four procedures for user calls
(see Section~\ref{galfeatures} for further features): 

\begin{enumerate}
\item The subroutine 
      {\tt \packagename\_initialize} 
      is used to set default values, and initialize private data, 
      before solving one or more problems with the
      same sparsity and bound structure.
\item The subroutine 
      {\tt \packagename\_analyse} 
      is called to analyze the sparsity pattern of the Hessian
      and to generate information that will be used when estimating 
      its values.
\item The subroutine 
      {\tt \packagename\_estimate} 
      is called repeatedly to estimate the Hessian by finite differences at 
      one or more given points.
\item The subroutine 
      {\tt \packagename\_terminate} 
      is provided to allow the user to automatically deallocate array 
       components of the private data, allocated by 
       {\tt \packagename\_solve}, 
       at the end of the solution process. 
       It is important to do this if the data object is re-used for another 
       problem {\bf with a different structure}
       since {\tt \packagename\_initialize} cannot test for this situation, 
       and any existing associated targets will subsequently become unreachable.
\end{enumerate}
We use square brackets {\tt [ ]} to indicate \optional\ arguments.

%%%%%% initialization subroutine %%%%%%

\subsubsection{The initialization subroutine}\label{subinit}
 Default values are provided as follows:
\vspace*{1mm}

\hspace{8mm}
{\tt CALL \packagename\_initialize( data, control, inform )}

\vspace*{-3mm}
\begin{description}

\itt{data} is a scalar \intentinout\ argument of type 
{\tt \packagename\_data\_type}
(see Section~\ref{typedata}). It is used to hold data about the problem being 
solved. 

\itt{control} is a scalar \intentout\ argument of type 
{\tt \packagename\_control\_type}
(see Section~\ref{typecontrol}). 
On exit, {\tt control} contains default values for the components as
described in Section~\ref{typecontrol}.
These values should only be changed after calling 
{\tt \packagename\_initialize}.

\itt{inform} is a scalar \intentout\ argument of type 
{\tt \packagename\_inform\_type}
(see Section~\ref{typeinform}). A successful call to
{\tt \packagename\_initialize}
is indicated when the  component {\tt status} has the value 0. 
For other return values of {\tt status}, see Section~\ref{galerrors}.

\end{description}

%%%%%%%%% analysis subroutine %%%%%%

\subsubsection{The analysis subroutine}
The analysis phase, in which the sparsity pattern of the Hessian
is used to generate information that will be used when estimating 
its values, is called as follows:

\vspace*{1mm}

\hspace{8mm}
{\tt CALL \packagename\_analyse( n, nz, ROW, DIAG, data, control, inform )}

\vspace*{-2mm}
\begin{description}
\ittf{n} 
is a scalar \intentin\ scalar argument of type \integer, that must be 
set to $n$ the dimension of the Hessian matrix, i.e. the number of
variables in the function $f$. 
\restrictions {\tt n} $> 0$.

\ittf{nz} 
is a scalar \intentin\ scalar argument of type \integer, that must be
set to the number of nonzero entries on and below the diagonal of the Hessian
matrix.
\restrictions {\tt n} $\le$ {\tt nz} $\le$ {\tt n} $\ast$ {\tt (n+1)/2}.

\ittf{ROW} is a scalar \intentinout\ rank-one array argument of type 
\integer\ and dimension {\tt nz}, that is used to describe 
the sparsity structure of the Hessian matrix. It must be set so that
 {\tt ROW(}$i${\tt)}, $i = 1, \ldots,$ {\tt nz} contain the row 
numbers of the successive nonzero elements of the {\bf lower triangular part
(including the diagonal)} of the Hessian matrix when scanned column after
column in the natural order. The diagonal entry 
{\bf must preceed the other entries} in each column. The remaining entries 
may appear in any order. On exit {\tt ROW} will be as input, but will have
been altered in the interim.
\restrictions 
$j \leq$ {\tt ROW(}$j${\tt)} $\leq n$ , $j = 1, \ldots,$ {\tt nz}.

\itt{DIAG} 
is a scalar \intentin\ rank-one array argument of type 
\integer\ and dimension {\tt n}, that is used to describe 
the sparsity structure of the Hessian matrix. It must be set so that
 {\tt DIAG(}$i${\tt)} $i$, $i = 1, \ldots,$ {\tt n}
contain the position of the $i$th diagonal of the matrix in the list held
in {\tt ROW}.
\restrictions {\tt ROW(DIAG(}$i${\tt))=} $i$, $i = 1, \ldots,$ {\tt n}.

\itt{control} is a scalar \intentin\ argument of type 
{\tt \packagename\_control\_type}
(see Section~\ref{typecontrol}). Default values may be assigned by calling 
{\tt \packagename\_initialize} prior to the first call to 
{\tt \packagename\_analyse}. 

\itt{inform} is a scalar \intentinout\ argument of type 
{\tt \packagename\_inform\_type}
(see Section~\ref{typeinform}). 
A successful call to
{\tt \packagename\_analyse}
is indicated when the  component {\tt status} has the value 0. 
For other return values of {\tt status}, see Section~\ref{galerrors}.

\itt{data} is a scalar \intentinout\ argument of type 
{\tt \packagename\_data\_type}
(see Section~\ref{typedata}). It is used to hold data about the problem being 
solved. With the possible exceptions of the components 
{\tt eval\_status} and {\tt U} (see Section~\ref{reverse}), 
it must not have been altered {\bf by the user} since the last call to 
{\tt \packagename\_initialize}.

\end{description}

%%%%%%%%% estimation subroutine %%%%%%

\subsubsection{The estimation subroutine}
The estimation phase, in which the nonzero entries of the Hessian
are estimated by finite differences, is called as follows:
\vspace*{1mm}

\hspace{8mm}
{\tt CALL \packagename\_estimate( n, nz, ROW, DIAG, X, G, STEPSIZE, H, \,\,\,\,\,\,   \& }
\vspace*{-1mm}

\hspace{41mm}
{\tt data, control, inform, userdata, eval\_G )}

\vspace*{-2mm}
\begin{description}
\itt{n,} {\tt nz}, {\tt ROW} and {\tt DIAG} are  \intentin\ arguments 
exactly as described and input to {\tt \packagename\_analyse},
and must not have been changed in the interim.

\ittf{X} is a scalar \intentin\ rank-one array argument of type 
\realdp, and dimension {\tt n}, that  must be set so that
{\tt X(}$i${\tt)} contains the component $x_i$, $i = 1, \ldots,$ {\tt n} 
of the point $\bmx$ at which the user wishes to estimate $\bmH(\bmx)$.

\ittf{G} is a scalar \intentin\ rank-one array argument of type 
\realdp, and dimension {\tt n}, that must be set so that
{\tt G(}$i${\tt)} contains the component $g_i(\bmx)$, $i = 1, \ldots,$ {\tt n} 
of the gradient $\bmg(\bmx)$ of $f$ at the point $\bmx$ input in {\tt X}.

\itt{STEPSIZE} is a scalar \intentin\ rank-one array argument of type 
\realdp, and dimension {\tt n}, that must be set to the stepsizes
to be used in the finite difference scheme.  One can roughly say that 
{\tt STEPSIZE(}$i${\tt)} is the step used to evaluate the $i$th column 
of the Hessian---recommended values are between $10^{-7}$ and $10^{-3}$
times the corresponding  component of {\tt X}.

\ittf{H} is a scalar \intentinout\ rank-one array argument of type 
\realdp, and dimension {\tt nz}, that needs not be set on input,
but that will be set to the non-zeros of the Hessian $\bmH(\bmx)$ in the
order defined by the list stored in {\tt ROW}.

\itt{control} is a scalar \intentin\ argument of type 
{\tt \packagename\_control\_type}
(see Section~\ref{typecontrol}). Default values may be assigned by calling 
{\tt \packagename\_initialize} prior to the first call to 
{\tt \packagename\_analyse}. 

\itt{inform} is a scalar \intentinout\ argument of type 
{\tt \packagename\_inform\_type}
(see Section~\ref{typeinform}). 
A successful call to
{\tt \packagename\_analyse}
is indicated when the  component {\tt status} has the value 0. 
For other return values of {\tt status}, see Sections~\ref{reverse} and
\ref{galerrors}.

\itt{data} is a scalar \intentinout\ argument of type 
{\tt \packagename\_data\_type}
(see Section~\ref{typedata}). It is used to hold data about the problem being 
solved. With the possible exceptions of the components 
{\tt eval\_status} and {\tt U} (see Section~\ref{reverse}), 
it must not have been altered {\bf by the user} since the last call to 
{\tt \packagename\_initialize}.

\itt{userdata} is a scalar \intentinout\ argument of type 
{\tt NLPT\_userdata\_type} whose components may be used
to communicate user-supplied data to and from the
\optional\ subroutine {\tt eval\_G},
(see Section~\ref{typeuserdata}). 

\itt{eval\_G} is an \optional\ 
user-supplied subroutine whose purpose is to evaluate the value of the 
gradient of the objective function $\bmg(\bmx) = \nabla_x f(\bmx)$ 
at a given vector $\bmx$.
See Section~\ref{gfv} for details.
If {\tt eval\_G} is present, 
it must be declared {\tt EXTERNAL} in the calling program.
If {\tt eval\_G} is absent, \solver\ will use reverse communication to
obtain gradient values (see Section~\ref{reverse}).

\end{description}

%%%%%%% termination subroutine %%%%%%

\subsubsection{The  termination subroutine}
All previously allocated arrays are deallocated as follows:
\vspace*{1mm}

\hspace{8mm}
{\tt CALL \packagename\_terminate( data, control, inform )}

\vspace*{-1mm}
\begin{description}

\itt{data} is a scalar \intentinout\ argument of type 
{\tt \packagename\_data\_type} 
exactly as for
{\tt \packagename\_solve},
which must not have been altered {\bf by the user} since the last call to 
{\tt \packagename\_initialize}.
On exit, array components will have been deallocated.

\itt{control} is a scalar \intentin\ argument of type 
{\tt \packagename\_control\_type}
exactly as for
{\tt \packagename\_solve}.

\itt{inform} is a scalar \intentout\ argument of type
{\tt \packagename\_inform\_type}
exactly as for
{\tt \packagename\_solve}.
Only the component {\tt status} will be set on exit, and a 
successful call to 
{\tt \packagename\_terminate}
is indicated when this  component {\tt status} has the value 0. 
For other return values of {\tt status}, see Section~\ref{galerrors}.

\end{description}

%%%%%%%%%%%%%%%%%%%%%% Gradient values %%%%%%%%%%%%%%%%%%%%%%

\subsubsection{Gradient values via internal evaluation\label{gfv}}

If the argument {\tt eval\_G} is present when calling \solver, the
user is expected to provide a subroutine of that name to evaluate the
value of the gradient the objective function $\bmg(\bmx) = \nabla_x f(\bmx)$.
The routine must be specified as

\def\baselinestretch{0.8}
{\tt \begin{verbatim}
       SUBROUTINE eval_G( status, X, userdata, G )
\end{verbatim} }
\def\baselinestretch{1.0}
\noindent whose arguments are as follows:

\begin{description}
\itt{status} is a scalar \intentout\ argument of type \integer,
that should be set to 0 if the routine has been able to evaluate
the gradient of the objective function
and to a non-zero value if the evaluation has not been possible.

\ittf{X} is a rank-one \intentin\ array argument of type \realdp\
whose components contain the vector $\bmx$.

\itt{userdata} is a scalar \intentinout\ argument of type 
{\tt NLPT\_userdata\_type} whose components may be used
to communicate user-supplied data to and from the subroutine {\tt eval\_G}
(see Section~\ref{typeuserdata}). 

\itt{G} is a rank-one \intentout\ argument of type \realdp,
whose components should be set to the values of the gradient 
of the objective function $\bmg(\bmx) = \nabla_x f(\bmx)$
evaluated at the vector $\bmx$ input in {\tt X}.

\end{description}

%%%%%%%%%%%%%%%%%%%%%% Reverse communication %%%%%%%%%%%%%%%%%%%%%%%%

\subsection{\label{reverse}Reverse Communication Information}

A positive value of {\tt inform\%status} on exit from 
{\tt \packagename\_estimate}
indicates that
\solver\ is seeking further information---this will happen 
if the user has chosen not to evaluate gradient
values internally (see Section~\ref{gfv}).
The user should compute the required information and re-enter \solver\
with {\tt inform\%status} and all other arguments (except those specifically
mentioned below) unchanged.

Possible values of {\tt inform\%status} and the information required are
\begin{description}
\ittf{1.} The user should compute the gradient 
     of the objective function $\bmg(\bmx) = \nabla_x f(\bmx)$ at the 
     point $\bmx$ indicated in {\tt data\%X}.
     The value of the $i$-th component of the gradient should be set 
     in {\tt data\%G(i)},      for $i = 1, \ldots, n$ and 
     {\tt data\%eval\_status} should be set to 0. If the user is
     unable to evaluate a component of $\bmg(\bmx)$---for instance, 
     if a component of the gradient is
     undefined at $\bmx$---the user need not set {\tt data\%G}, but
     should then set {\tt data\%eval\_status} to a non-zero value.
\end{description}


%%%%%%%%%%%%%%%%%%%%%% Warning and error messages %%%%%%%%%%%%%%%%%%%%%%%%

\galerrors
A negative value of {\tt inform\%status} on exit from 
{\tt \packagename\_analyse},
{\tt \packagename\_estimate}
or 
{\tt \packagename\_terminate}
indicates that an error has occurred. No further calls should be made
until the error has been corrected. Possible values are:

\begin{description}

\itt{\galerrallocate.} An allocation error occurred. 
A message indicating the offending
array is written on unit {\tt control\%error}, and the returned allocation 
status and a string containing the name of the offending array
are held in {\tt inform\%alloc\_\-status}
and {\tt inform\%bad\_alloc}, respectively.

\itt{\galerrdeallocate.} A deallocation error occurred. 
A message indicating the offending 
array is written on unit {\tt control\%error} and the returned allocation 
status and a string containing the name of the offending array
are held in {\tt inform\%alloc\_\-status}
and {\tt inform\%bad\_alloc}, respectively.

\itt{\galerrrestrictions.} 
 One or more of the restriction 
$0 <$ {\tt n} $\le$ {\tt nz} $\le$ {\tt n} $\ast$ {\tt (n+1)/2}
  has been violated.

\itt{\galerrupperentry.} One or more of the restrictions
{\tt ROW(DIAG(} $i$ {)\tt)=} $i$, $i = 1, \ldots,$ {\tt n}, or
$j \leq$ {\tt ROW(}$j${\tt)} $\leq n$ , $j = 1, \ldots,$ {\tt nz},
  has been violated. See {\tt inform\%bad\_row} 
for the index of the row involved.

\itt{\galerrcallorder.}  {\tt \packagename\_estimate} has been called
before {\tt \packagename\_analyse}. 

\end{description}

%%%%%%%%%%%%%%%%%%%%%% Further features %%%%%%%%%%%%%%%%%%%%%%%%

\galfeatures
\noindent In this section, we describe an alternative means of setting 
control parameters, that is components of the variable {\tt control} of type
{\tt \packagename\_control\_type}
(see Section~\ref{typecontrol}), 
by reading an appropriate data specification file using the
subroutine {\tt \packagename\_read\_specfile}. This facility
is useful as it allows a user to change  {\tt \packagename} control parameters 
without editing and recompiling programs that call {\tt \packagename}.

A specification file, or specfile, is a data file containing a number of 
"specification commands". Each command occurs on a separate line, 
and comprises a "keyword", 
which is a string (in a close-to-natural language) used to identify a 
control parameter, and 
an (optional) "value", which defines the value to be assigned to the given
control parameter. All keywords and values are case insensitive, 
keywords may be preceded by one or more blanks but
values must not contain blanks, and
each value must be separated from its keyword by at least one blank.
Values must not contain more than 30 characters, and 
each line of the specfile is limited to 80 characters,
including the blanks separating keyword and value.

The portion of the specification file used by 
{\tt \packagename\_read\_specfile}
must start
with a "{\tt BEGIN \packagename}" command and end with an 
"{\tt END}" command.  The syntax of the specfile is thus defined as follows:
\begin{verbatim}
  ( .. lines ignored by FDH_read_specfile .. )
    BEGIN FDH
       keyword    value
       .......    .....
       keyword    value
    END 
  ( .. lines ignored by FDH_read_specfile .. )
\end{verbatim}
where keyword and value are two strings separated by (at least) one blank.
The ``{\tt BEGIN \packagename}'' and ``{\tt END}'' delimiter command lines 
may contain additional (trailing) strings so long as such strings are 
separated by one or more blanks, so that lines such as
\begin{verbatim}
    BEGIN FDH SPECIFICATION
\end{verbatim}
and
\begin{verbatim}
    END FDH SPECIFICATION
\end{verbatim}
are acceptable. Furthermore, 
between the
``{\tt BEGIN \packagename}'' and ``{\tt END}'' delimiters,
specification commands may occur in any order.  Blank lines and
lines whose first non-blank character is {\tt !} or {\tt *} are ignored. 
The content 
of a line after a {\tt !} or {\tt *} character is also 
ignored (as is the {\tt !} or {\tt *}
character itself). This provides an easy manner to "comment out" some 
specification commands, or to comment specific values 
of certain control parameters.  

The value of a control parameters may be of three different types, namely
integer, logical or real.
Integer and real values may be expressed in any relevant Fortran integer and
floating-point formats (respectively). Permitted values for logical
parameters are "{\tt ON}", "{\tt TRUE}", "{\tt .TRUE.}", "{\tt T}", 
"{\tt YES}", "{\tt Y}", or "{\tt OFF}", "{\tt NO}",
"{\tt N}", "{\tt FALSE}", "{\tt .FALSE.}" and "{\tt F}". 
Empty values are also allowed for 
logical control parameters, and are interpreted as "{\tt TRUE}".  

The specification file must be open for 
input when {\tt \packagename\_read\_specfile}
is called, and the associated device number 
passed to the routine in device (see below). 
Note that the corresponding 
file is {\tt REWIND}ed, which makes it possible to combine the specifications 
for more than one program/routine.  For the same reason, the file is not
closed by {\tt \packagename\_read\_specfile}.

\subsubsection{To read control parameters from a specification file}
\label{readspec}

Control parameters may be read from a file as follows:
\hskip0.5in 
\def\baselinestretch{0.8} {\tt \begin{verbatim}
     CALL FDH_read_specfile( control, device )
\end{verbatim}
}
\def\baselinestretch{1.0}

\begin{description}
\itt{control} is a scalar \intentinout argument of type 
{\tt \packagename\_control\_type}
(see Section~\ref{typecontrol}). 
Default values should have already been set, perhaps by calling 
{\tt \packagename\_initialize}.
On exit, individual components of {\tt control} may have been changed
according to the commands found in the specfile. Specfile commands and 
the component (see Section~\ref{typecontrol}) of {\tt control} 
that each affects are given in Table~\ref{specfile}.

\bctable{|l|l|l|} 
\hline
  command & component of {\tt control} & value type \\ 
\hline
  {\tt error-printout-device} & {\tt \%error} & integer \\
  {\tt printout-device} & {\tt \%out} & integer \\
  {\tt print-level} & {\tt \%print\_level} & integer \\
  {\tt space-critical}   & {\tt \%space\_critical} & logical \\
  {\tt deallocate-error-fatal}   & {\tt \%deallocate\_error\_fatal} & logical \\
  {\tt output-line-prefix} & {\tt \%prefix} & character \\
\hline

\ectable{\label{specfile}Specfile commands and associated 
components of {\tt control}.}

\itt{device} is a scalar \intentin argument of type \integer,
that must be set to the unit number on which the specfile
has been opened. If {\tt device} is not open, {\tt control} will
not be altered and execution will continue, but an error message
will be printed on unit {\tt control\%error}.

\end{description}

%%%%%%%%%%%%%%%%%%%%%% Information printed %%%%%%%%%%%%%%%%%%%%%%%%

\galinfo
If {\tt control\%print\_level} is positive, information about 
errors encountered will be printed on unit {\tt control\-\%out}.

%%%%%%%%%%%%%%%%%%%%%% GENERAL INFORMATION %%%%%%%%%%%%%%%%%%%%%%%%

\galgeneral

\galcommon None.
\galworkspace Provided automatically by the module.
\galroutines None. 
\galmodules {\tt \packagename\_solve} calls the \galahad\ packages
{\tt GALAHAD\_SY\-M\-BOLS}, 
{\tt GALAHAD\_SPECFILE},
{\tt GALAHAD\_SPACE} and
{\tt GALAHAD\_NLPT}.
\galio Output is under control of the arguments
 {\tt control\%error}, {\tt control\%out} and {\tt control\%print\_level}.
\galrestrictions 
$0 <$ {\tt n} $\le$ {\tt nz} $\le$ {\tt n} $\ast$ {\tt (n+1)/2},
{\tt ROW(DIAG(} $i$ {)\tt)=} $i$, $i = 1, \ldots,$ {\tt n}, or
$j \leq$ {\tt ROW(}$j${\tt)} $\leq n$ , $j = 1, \ldots,$ {\tt nz}.
\galportability ISO Fortran~95 + TR 15581 or Fortran~2003. 
The package is thread-safe.

%%%%%%%%%%%%%%%%%%%%%% METHOD %%%%%%%%%%%%%%%%%%%%%%%%

\galmethod
The routines  use a ``Lower triangular  substitution
algorithm''. It assumes  that no  diagonal
values  are constrained  by the sparsity  requirements.
The analysis phase uses the  sparsity pattern  of
the matrix  to decide  how many differences  in
gradient are needed  for estimation  and along what directions.
For this purpose,  it defines  a symmetric permutation  of the
matrix. The evaluation phase  computes  the differences  in
gradients  that are required  and then  solves  a consequent
linear  system by a  substitution to obtain  the entries of  the
approximate  Hessian. 

Once the  pattern analysis  is performed,  one can
approximate  the Hessian  of $f$ at several different points.
This is done  by a single call  to {\tt \packagename\_analyse}
followed  by several calls to {\tt \packagename\_estimate}
for different values of $x$ with corresponding $g(x)$.

\vspace*{1mm}

\galreference
\vspace*{1mm}

\noindent
The method is described in detail in
\vspace*{1mm}

\noindent
M.\ J.\ D.\ Powell and  Ph.\ L.\ Toint. ``On the estimation  of
sparse Hessian matrices''. SIAM J. Numer. Anal. {\bf 16} (1979) 1060-1074.

%%%%%%%%%%%%%%%%%%%%%% EXAMPLE %%%%%%%%%%%%%%%%%%%%%%%%

\galexamples
Suppose we wish to estimate the Hessian matrix of the objective function 
\disp{f(x_1 , \ldots, x_5 ) =  ( x_1 + p )^3 + x_2^3 + x_3^3 + x_4^3 + x_5^3 
 + x_1^{} x_4^{} + x_2^{} x_3^{} + x_3^{} x_4^{} + x_4^{} x_5^{}, } 
that depends on the parameter $p$, whose gradient is
\disp{\bmg(\bmx) = \vect{
3(x_1^{} + p )^2 + x_4^{} \\ 3x_2^2 + x_3^{} \\ 3x_3^2 + x_2^{} + x_4^{} \\
3x_4^2 + x_1^{} + x_3^{} + x_5^{} \\ 3x_5^2 + x_4^{}
}}
and thus whose Hessian has the sparsity pattern
\disp{ \mat{ccccc}{ 
\ast & 0 & 0 & \cdot & 0 \\
0 & \ast & \cdot & 0 & 0 \\
0 & \ast & \ast & \cdot & 0 \\
\ast & 0 & \ast & \ast & \cdot \\
0 & 0 & 0 & \ast & \ast}}
(the entries $\ast$ in the lower triangle are the ones that will be estimated).

Choosing uniform  step lengths of size $10^{-6}$, we may estimate the
Hessian at $x = (1,1,1,1,1)^T$ and $(1,2,3,4,5)^T$ for $p = 4$ using the
following code - we use an explicit call to evaluate the gradient 
at the first point and the reverse-communication method at the second:

{\tt \small
\VerbatimInput{\packageexample}
}
\noindent
Notice how the parameter $p$ is passed to the function evaluation 
routines via the {\tt real} component of the derived type {\tt userdata}.
The code produces the following output:
{\tt \small
\VerbatimInput{\packageresults}
}
\noindent

\end{document}

