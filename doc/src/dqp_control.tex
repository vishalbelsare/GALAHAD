\begin{description}

\itt{error} is a scalar variable of type \integer, that holds the
stream number for error messages. Printing of error messages in
{\tt \packagename\_solve} and {\tt \packagename\_terminate} is suppressed if
{\tt error} $\leq 0$.
The default is {\tt error = 6}.

\ittf{out} is a scalar variable of type \integer, that holds the
stream number for informational messages. Printing of informational messages in
{\tt \packagename\_solve} is suppressed if {\tt out} $< 0$.
The default is {\tt out = 6}.

\itt{print\_level} is a scalar variable of type \integer, that is used
to control the amount of informational output which is required. No
informational output will occur if {\tt print\_level} $\leq 0$. If
{\tt print\_level} $= 1$, a single line of output will be produced for each
iteration of the process. If {\tt print\_level} $\geq 2$, this output will be
increased to provide significant detail of each iteration.
The default is {\tt print\_level = 0}.

\itt{start\_print} is a scalar variable of type \integer, that specifies
the first iteration for which printing will occur in {\tt \packagename\_solve}.
If {\tt start\_print} is negative, printing will occur from the outset.
The default is {\tt start\_print = -1}.

\itt{stop\_print} is a scalar variable of type \integer, that specifies
the last iteration for which printing will occur in  {\tt \packagename\_solve}.
If {\tt stop\_print} is negative, printing will occur once it has been
started by {\tt start\_print}.
The default is {\tt stop\_print = -1}.

\itt{print\_gap} is a scalar variable of type \integer.
Once printing has been started, output will occur once every
{\tt print\_gap} iterations. If {\tt print\_gap} is no larger than 1,
printing will be permitted on every iteration.
The default is {\tt print\_gap = 1}.

\itt{maxit} is a scalar variable of type \integer, that holds the
maximum number of iterations which will be allowed in {\tt \packagename\_solve}.
The default is {\tt maxit = 1000}.

\itt{max\_sc} is a scalar variable of type \integer, that specifies
the maximum number of columns permitted in the Schur complement of the
reference matrix (see Section~\ref{galmethod})
before a refactorization is triggered when there is no Fredholm Alternative.
The default is {\tt max\_sc = 100}.

\itt{cauchy\_only} is a scalar variable of type \integer, that specifies
the maximum number of changes in the active set that may occur in the
first-phase projected-dual-gradient arc search during an iteration before
attempting a second phase unconstrained minimization in the space of
free dual variables \S~\ref{galmethod}). If {\tt cauchy\_only} is negative,
the second phase will always be tried.
The default is {\tt cauchy\_only = -1}.

\itt{arc\_search\_maxit} is a scalar variable of type \integer,
that holds the maximum number of steps that may be performed
by the arc-search every iteration.
If {\tt arc\_search\_maxit} is set to a negative number, as many steps
as are necessary will be performed.
The default is {\tt arc\_search\_maxit = -1}.

\itt{cg\_maxit} is a scalar variable of type \integer, that holds the
maximum number of conjugate-gradient inner iterations that may be performed
during the computation of each search direction in {\tt DQP\_solve}.
If {\tt cg\_maxit} is set to a negative number, it will be reset by
{\tt DQP\_solve} to the dimension of the relevant linear system $+ 1$.
The default is {\tt cg\_maxit = 1000}.
%The default is {\tt cg\_maxit = - 1}.

\itt{dual\_starting\_point} is a scalar variable of type \integer, that
specifies how the algorithm computes its staring point.
Possible values are:
\begin{description}
\itt{0} the values $\bmy$ and $\bmz$
        provided by the user in components {\tt Y} and {\tt Z}
        of the derived data type {\tt QPT\_problem\_type} will be used.
\itt{1} values obtained by minimizing a linearized version of the dual
        will be used.
\itt{2} values obtained by minimizing a simplified quadratic version of
        the dual will be used.
\itt{3} values will be chosen so that all dual variables lie away from
        their bounds if possible. This corresponds to trying to start
        from a point in which all primal constraints are active.
\itt{4} values will be chosen so that all dual variables lie on
        their bounds if possible. This corresponds to trying to start
        from a point in which all primal constraints are inactive.
\end{description}
Any other value will be interpreted as {\tt dual\_starting\_point = 0},
and this is the default.

\ifthenelse{\equal{\package}{dqp}}{
\itt{restore\_problem} is a scalar variable of type \integer, that
specifies how much of the input problem is to be restored on output.
Possible values are:
\begin{description}
\itt{0} nothing is restored.
\itt{1} the vector data $\bmw$, $\bmg$,
   $\bmc^{l}$, $\bmc^{u}$, $\bmx^{l}$, and $\bmx^{u}$
   will be restored to their input values.
\itt{2} the entire problem, that is the above vector data along with
the Jacobian matrix $\bmA$, will be restored.
\end{description}
The default is {\tt restore\_problem = 2}.
}{}

\itt{rho} is a scalar variable of type \realdp, that holds
the penalty weight, $\rho$. If {\tt rho > 0}, the general linear constraints are
not enforced explicitly, but instead included in the objective as a penalty term
weighted by $\rho$. If {\tt rho} $\leq 0$, the general linear constraints are
explicit (that is, there is no penalty term in the objective function)
The default is {\tt rho =} $0.0$.

\itt{infinity} is a scalar variable of type \realdp, that is used to
specify which constraint bounds are infinite.
Any bound larger than {\tt infinity} in modulus will be regarded as infinite.
The default is {\tt infinity =} $10^{19}$.

\itt{stop\_abs\_p} and {\tt stop\_rel\_p}
are scalar variables of type \realdp, that hold the
required absolute and relative accuracy for the primal infeasibility
(see Section~\ref{galmethod}).
The absolute value of each component of the primal infeasibility
on exit is required to be smaller than the larger of {\tt stop\_abs\_p} and
{\tt stop\_rel\_p} times a ``typical value'' for this component.
The defaults are {\tt stop\_abs\_p =} {\tt stop\_rel\_p =} $u^{1/3}$,
where $u$ is {\tt EPSILON(1.0)} ({\tt EPSILON(1.0D0)} in
{\tt \fullpackagename\_double}).

\itt{stop\_abs\_d} and {\tt stop\_rel\_d}
are scalar variables of type \realdp, that hold the
required absolute and relative accuracy for the dual infeasibility
(see Section~\ref{galmethod}).
The absolute value of each component of the dual infeasibility
on exit is required to be smaller than the larger of {\tt stop\_abs\_p} and
{\tt stop\_rel\_p} times a ``typical value'' for this component.
The defaults are {\tt stop\_abs\_d =} {\tt stop\_rel\_d =} $u^{1/3}$,
where $u$ is {\tt EPSILON(1.0)} ({\tt EPSILON(1.0D0)} in
{\tt \fullpackagename\_double}).

\itt{stop\_abs\_c} and {\tt stop\_rel\_c}
are scalar variables of type \realdp, that hold the
required absolute and relative accuracy
for the violation of complementary slackness
(see Section~\ref{galmethod}).
The absolute value of each component of the complementary slackness
on exit is required to be smaller than the larger of {\tt stop\_abs\_p} and
{\tt stop\_rel\_p} times a ``typical value'' for this component.
The defaults are {\tt stop\_abs\_c =} {\tt stop\_rel\_c =} $u^{1/3}$,
where $u$ is {\tt EPSILON(1.0)} ({\tt EPSILON(1.0D0)} in
{\tt \fullpackagename\_double}).

\itt{stop\_cg\_relative} and {\tt stop\_cg\_absolute}
are scalar variables of type \realdp,
that hold the relative and absolute convergence tolerances for the
conjugate-gradient iteration that occurs in the face of currently-active
constraints that may be used to construct the search direction.
{\tt \_stop\_cg\_relative = 0.01}
and \sloppy {\tt stop\_cg\_absolute =} $\sqrt{u}$,
where $u$ is {\tt EPSILON(1.0)} ({\tt EPSILON(1.0D0)} in
{\tt \fullpackagename\_double}).

\itt{cg\_zero\_curvature} is a scalar variable of type \realdp
that specifies the threshold below which any curvature
encountered by the conjugate-gradient iteration is regarded as zero.
The default is {\tt cg\_zero\_curvature =} $10 u$,
where $u$ is {\tt EPSILON(1.0)} ({\tt EPSILON(1.0D0)} in
{\tt \fullpackagename\_double}).

\itt{identical\_bounds\_tol}
is a scalar variable of type \realdp.
Each pair of variable bounds $(x_{j}^{l}, x_{j}^{u})$
that is closer than {\tt identical\_bounds\_tol}
will be reset to the average of their values,
$\half (x_{j}^{l} + x_{j}^{u})$.
The default is {\tt identical\_bounds\_tol =} $u$,
where $u$ is {\tt EPSILON(1.0)} ({\tt EPSILON(1.0D0)} in
{\tt \fullpackagename\_double}).

\itt{cpu\_time\_limit} is a scalar variable of type \realdp,
that is used to specify the maximum permitted CPU time. Any negative
value indicates no limit will be imposed. The default is
{\tt cpu\_time\_limit = - 1.0}.

\itt{clock\_time\_limit} is a scalar variable of type \realdp,
that is used to specify the maximum permitted elapsed system clock time.
Any negative value indicates no limit will be imposed. The default is
{\tt clock\_time\_limit = - 1.0}.

\itt{remove\_dependencies} is a scalar variable of type
default \logical, that must be set \true\ if the algorithm
is to attempt to remove any linearly dependent constraints before
solving the problem, and \false\ otherwise.
We recommend removing linearly dependencies.
The default is {\tt remove\_dependencies = .TRUE.}.

\itt{treat\_zero\_bounds\_as\_general} is a scalar variable of type
default \logical.
If it is set to \false, variables which
are only bounded on one side, and whose bound is zero,
will be recognised as non-negativities/non-positivities rather than simply as
lower- or upper-bounded variables.
If it is set to \true, any variable bound
$x_{j}^{l}$ or $x_{j}^{u}$ which has the value 0.0 will be
treated as if it had a general value.
Setting {\tt treat\_zero\_bounds\_as\_general} to \true\ has the advantage
that if a sequence of problems are reordered, then bounds which are
``accidentally'' zero will be considered to have the same structure as
those which are nonzero. However, {\tt \fullpackagename} is
able to take special advantage of non-negativities/non-positivities, so
if a single problem, or if a sequence of problems whose
bound structure is known not to change, is/are to be solved,
it will pay to set the variable to \false.
The default is {\tt treat\_zero\_bounds\_as\_general = .FALSE.}.

\itt{exact\_arc\_search} is a scalar variable of type default \logical, that
must be set \true\ if the user wishes to perform an exact arc search
and \false\ if an inexact search suffices. Usually the exact search is
beneficial, but occasionally it may be more expensive.
The default is {\tt exact\_arc\_search = .TRUE.}.

\itt{subspace\_direct} is a scalar variable of type default \logical, that
must be set \true\ if the user wishes to compute subspace steps using
matrix factorization,
and \false\ if conjugate-gradient steps are preferred. Factorization
often produces a better step, but sometimes the conjugate-gradient method
may be less expensive and less demanding on storage.
The default is {\tt subspace\_direct = .FALSE.}, but if more advanced
symmetric linear solvers such as {\tt MA57} or {\tt MA97} available,
we recommend setting {\tt subspace\_direct = .TRUE.} and changing
{\tt symmetric\_linear\_solver} (see below) appropriately.

\itt{subspace\_arc\_search} is a scalar variable of type default \logical, that
must be set \true\ if the user wishes to perform an arc search following
the subspace step and \false\ if a step to the nearest inactive bound suffices.
As before, the exact search is usually beneficial, but it is more expensive.
The default is {\tt subspace\_arc\_search = .TRUE.}.

\itt{space\_critical} is a scalar variable of type default \logical,
that must be set \true\ if space is critical when allocating arrays
and  \false\ otherwise. The package may run faster if
{\tt space\_critical} is \false\ but at the possible expense of a larger
storage requirement. The default is {\tt space\_critical = .FALSE.}.

\itt{deallocate\_error\_fatal} is a scalar variable of type default \logical,
that must be set \true\ if the user wishes to terminate execution if
a deallocation  fails, and \false\ if an attempt to continue
will be made. The default is {\tt deallocate\_error\_fatal = .FALSE.}.

\itt{symmetric\_linear\_solver} is a scalar variable of type default \character\
and length 30, that specifies the external package to be used to
solve any symmetric linear system that might arise. Current possible
choices are {\tt 'sils'}, {\tt 'ma27'}, {\tt 'ma57'}, {\tt 'ma77'},
{\tt 'ma86'}, {\tt 'ma97'}, {\tt ssids}, {\tt 'pardiso'}
and {\tt 'wsmp'}.
See the documentation for the \galahad\ package {\tt SLS} for further details.
Since  {\tt 'sils'} does not currently provide the required Fredholm
Alternative option, the default is {\tt symmetric\_linear\_solver = 'ma57'},
but we recommend {\tt 'ma97'} if it is available.

\itt{definite\_linear\_solver} is a scalar variable of type default \character\
and length 30, that specifies the external package to be used to
solve any symmetric positive-definite linear system that might arise.
Current possible
choices are {\tt 'sils'}, {\tt 'ma27'}, {\tt 'ma57'}, {\tt 'ma77'},
{\tt 'ma86'}, {\tt 'ma87'}, {\tt 'ma97'}, {\tt ssids}, {\tt 'pardiso'}
and {\tt 'wsmp'}.
See the documentation for the \galahad\ package {\tt SLS} for further details.
The default is {\tt definite\_linear\_solver = 'sils'},
but we recommend {\tt 'ma87'} if it available.

\itt{unsymmetric\_linear\_solver} is a scalar variable of type default
\character\
and length 30, that specifies the external package to be used to
solve any unsymmetric linear systems that might arise. Possible
choices are
{\tt 'gls'},
{\tt 'ma28'}
and
{\tt 'ma48'},
although only {\tt 'gls'} is installed by default.
See the documentation for the \galahad\ package {\tt ULS} for further details.
The default is {\tt unsymmetric\_linear\_solver = 'gls'},
but we recommend {\tt 'ma48'} if it available.

\itt{prefix} is a scalar variable of type default \character\
and length 30, that may be used to provide a user-selected
character string to preface every line of printed output.
Specifically, each line of output will be prefaced by the string
{\tt prefix(2:LEN(TRIM(prefix))-1)},
thus ignoring the first and last non-null components of the
supplied string. If the user does not want to preface lines by such
a string, they may use the default {\tt prefix = ""}.

\itt{FDC\_control} is a scalar variable of type
{\tt FDC\_control\_type}
whose components are used to control any detection of linear dependencies
performed by the package
{\tt \libraryname\_FDC}.
See the specification sheet for the package
{\tt \libraryname\_FDC}
for details, and appropriate default values.

\itt{SLS\_control} is a scalar variable of type
{\tt SLS\_control\_type}
whose components are used to control factorizations
performed by the package
{\tt \libraryname\_SLS}.
See the specification sheet for the package
{\tt \libraryname\_SLS}
for details, and appropriate default values.

\itt{SBLS\_control} is a scalar variable of type
{\tt SBLS\_control\_type}
whose components are used to control factorizations
performed by the package
{\tt \libraryname\_SBLS}.
See the specification sheet for the package
{\tt \libraryname\_SBLS}
for details, and appropriate default values.

\itt{GLTR\_control} is a scalar variable of type
{\tt GLTR\_control\_type}
whose components are used to control conjugate-gradient solves
performed by the package
{\tt \libraryname\_GLTR}.
See the specification sheet for the package
{\tt \libraryname\_GLTR}
for details, and appropriate default values.

\end{description}
